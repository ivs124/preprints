\documentclass[a4paper]{article}
\usepackage{a4wide}
% \usepackage{graphicx}
% \usepackage{amssymb,amsmath}
% \usepackage{booktabs}
\usepackage[english]{babel}
% \usepackage{multirow}
\usepackage{graphicx}% Include figure files
% \usepackage{color}
\emergencystretch=35pt
\usepackage[pdftex,unicode,colorlinks, citecolor=blue,%
filecolor=black, linkcolor=blue, urlcolor=black]{hyperref}
\usepackage[figure,table]{hypcap}


\begin{document}
\begin{center}
  Response Letter
\end{center}
Dear Editor,
\\
\\
I am writing to respond to the reviewers' critical comments on our
manuscript entitled, ``Superabsorption of light by nanoparticles''
NR-COM-08-2015-005468, which we aim for publication in Nanoscale.

I would like to thank the editor and two anonymous reviewers for their
constructive comments, which helped us to improve the
manuscript. Below, we address all comments point-by-point, discussing
the subsequent modifications.

\vspace{10pt}

\textbf{Reviewer \#1 comments}

\begin{tabular}[!H]{l|p{0.9\textwidth}}
\quad & 1. The authors claimed “Combined effect of these resonances is presented to produce the flat and relative broadband electric resonance response.” Nevertheless, the broadened absorption band is not very broad. Is there any further way to predict a broadband light absorption. For instance, a broadband absorption in the whole visible spectral range. Maybe, a possible way of using the dispersed size scale nanoparticles should be added for improving the study. 
\end{tabular}

400-700 3-layer
4-layer?

В статье этого графика нет, есть обсуждение, готовим новую статью.

In order to reply to this comment we run an optimization ... No changes
to the manuscript were made.


\begin{tabular}[!H]{l|p{0.9\textwidth}}
  \quad & 2. To achieve super-absorption behavior, it is interesting
  to know what will happen when the multilayered nanoparticles are
  closely packed as the plasmonic crystal. As reported in the previous
  papers [ACS Applied Materials \& Interfaces, 7, 4962−4968 (2015);
  Materials Letters 158, 262–265 (2015); Applied Physics Letters, 104,
  081116 (2014); Nanotechnology, 24, 155203 (2013)], the packed
  plasmonic crystals have been demonstrated to show broadband light
  coupling and confinement. Thereby, it would be interesting to show
  improved broadband light absorption based on the plasmonic crystal
  of this proposed multilayered nanoparticles. 
\end{tabular}

Lumerical - periodic boundary - попробовать несколько частиц



We totally agree with the reviewer's statement. It seems that our
manuscript was not clear enough at this point. ...  However, as we
compare final optimization results we conclude that .... To make it
clearer we rewrite this part of the manuscript as follows:

\begin{tabular}[!H]{p{0.9\textwidth}}
`` Some text''
\end{tabular}

It should be mentioned that .... No changes to the
manuscript were made.

\vspace{10pt}

\begin{minipage}{1.0\linewidth}
  \textbf{Reviewer \#2 comments}\\
  \begin{tabular}[!H]{l|p{0.9\textwidth}}
    \quad & 1.  As we can find from the manuscript, the highest
    absorption efficiency is achieved for a Si/Ag core-shell
    structure, which is not located in the super absorbing
    regime. Moreover, the authors also claimed that from practical
    aspect, the core-shell structure (not in the super absorbing
    regime) could be easier and cheaper to fabricate than three
    layered structure (in the super absorbing regime). Therefore, the
    authors should clearly clarify what are the advantages or
    significances of the super absorption nanoparticles? 
\end{tabular}
\end{minipage}

Для маленьких толщин лучше простой диполь, если есть ограничения на
минимальный размер частицы - лучше суперпоглотитель. Или, например,
если важно ещё и абсолютное поглощение от одиночной частицы - большая
частица поглощает больше, для неё оптимальнее работа в супер- режиме. 


There are several points to be treated carefully when discussing...
 However, to claim this concept to be a general one ...  more simulations
should be performed.

\begin{tabular}[!H]{l|p{0.9\textwidth}}
  \quad &  2.      In Fig. 2, we noticed a discontinuity at ~80
  nm. The authors explain it as the design supporting electric dipole
  and magnetic quadrupole has larger ACS. However, this explanation is
  not clearly to me since it is lacking physics behind this
  phenomenon. The authors should clarify why the magnetic quadrupole
  only plays a significant role in this small wavelength range. 
\end{tabular}



\begin{tabular}[!H]{l|p{0.9\textwidth}}
\quad & 3.      In Fig. 3 (c), the authors observed a flat top of
electric dipole resonance. They attributed this flat resonance to the
excited several electric dipole resonances with close resonance
frequencies. Nevertheless, as we can see in Fig. 3 (d), even without
considering the resonances located in outer and inner shell, the
resonance inside the core is much broader than the other two
cases. The authors should explain this broadened resonance clearly. 
\end{tabular}

Да, он прав, основной вклад от ядра, остальные делают верхушку
плоской. Фишка в том что резонанс плоский, а не широкий.

\begin{tabular}[!H]{l|p{0.9\textwidth}}
\quad & 4.      Some sentences are not clear to me. For instance,
“there is a strong conterplay between the increased absorption for
larger particles vs size for smaller particles”. 
In summary, I do not think the manuscript is acceptable at its current
stage. 
\end{tabular}

Илья

\\
\vspace{10pt}
\\
Sincerely Yours,\\
On behalf of the authors,\\
Konstantin Ladutenko
\end{document}
