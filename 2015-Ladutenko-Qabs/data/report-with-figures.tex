%%\documentclass[fullscreen=true, handout]{beamer}
\documentclass[fullscreen=true]{beamer}
\usetheme[secheader]{Boadilla}
%%\usetheme{Boadilla}
%%\usetheme[height=7mm]{Rochester} 
%%\useinnertheme{rounded}
 \usecolortheme{seahorse}
%% \usecolortheme{whale}
%% \usecolortheme{rose}
%%\setbeamertemplate{navigation symbols}{\insertframenavigationsymbol}
%% or
\beamertemplatenavigationsymbolsempty
%% \setbeamertemplate[info line]{footline} %%Не работает
%% {\quad\strut\insertsection
%% \hfill\insertframenumber/\inserttotalframenumber\strut\quad} 
\usepackage[utf8]{inputenc}
\usepackage[T1]{fontenc} %%apt-get install cm-super
\usepackage[english, russian]{babel}
%%\usepackage[english]{babel}
\emergencystretch=35pt %%Чтобы нe было строчек, вылезших на поля
\usepackage{ragged2e} %% to use \justifying
\usepackage{amsmath}
\usepackage{booktabs}
\usepackage{multicol}
\usepackage{listings}
%% \lstset{language=C++,
%%   backgroundcolor=\color{Black},
%%    basicstyle=\color{White}\tiny\ttfamily,
%%    keywordstyle=\color{Orange},
%%    identifierstyle=\color{Cyan},
%%    stringstyle=\color{Red}, 
%%    commentstyle=\color{Green},  
%%    breaklines=true,
%%    breakatwhitespace=true,
%%    tabsize=10,
%%    showstringspaces=false%%
%% }
%%or better
\lstset{breakatwhitespace,
language=C++,
columns=fullflexible,
keepspaces,
breaklines,
tabsize=2,
backgroundcolor=\color{white},
basicstyle=\color{black}\small\ttfamily,
keywordstyle=\color{orange},
identifierstyle=\color{blue},
stringstyle=\color{red}, 
commentstyle=\color{green},  
showstringspaces=false,
extendedchars=true}
\newcommand{\transhi}{40} %% Переменная - хорошая прозрачность
\setbeamercovered{transparent = \transhi} 
\setbeamertemplate{caption}[numbered]
\begin{document}
%% ----------------------------------------------------------------
\title{Report \today}
%% ----------------------------------------------------------------
\section{Figures}

\subsection{01Eps-re.pdf}
\begin{frame}
  \begin{figure}
    \includegraphics[height=0.8\textheight]{fig/01Eps-re.pdf}%
    \caption{Проблема с недостатоком разрешения - мало точек для
      материальных параметров.}
  \end{figure}
\end{frame}

\subsection{02Eps-im.pdf}
\begin{frame}
  \begin{figure}
    \includegraphics[height=0.8\textheight]{fig/02Eps-im.pdf}%
    \caption{Особено это хорошо заметно для мнимой части. Сделал
      кубическую интерполяцию}
  \end{figure}
\end{frame}

\subsection{03-SiAgSi-ab-spectra4.pdf}
\begin{frame}
  \begin{figure}
    \includegraphics[height=0.8\textheight]{fig/03-SiAgSi-ab-spectra4.pdf}%
    \caption{Все кривые стали глаже, этот рисунок вставил в статью.}
  \end{figure}
\end{frame}

\subsection{04SiAgSi3-flow-R81-YZ-Eabs-13.pdf}
\begin{frame}
  \begin{figure}
    \includegraphics[height=0.8\textheight]{fig/04SiAgSi3-flow-R81-YZ-Eabs-13.pdf}%
    \caption{Картинки полей для
      обоих случаев очень похоже - здесь нарисован |E|. Немного
      отличается величина усиления поля для разных длин волн, сильно
      отличается то, как рисуются линии потока энергии.}
  \end{figure}
\end{frame}

\subsection{05SiAgSi3-flow-R81-YZ-Eabs-39.pdf}
\begin{frame}
  \begin{figure}
    \includegraphics[height=0.8\textheight]{fig/05SiAgSi3-flow-R81-YZ-Eabs-39.pdf}%
    \caption{Более плотное покрытие линиями потока энергии. Ничего дополнительного для
      меня к картинке с коэффициентами по слоям не добавляют, поэтому
      пока их в рукопись не добавлял.
}
  \end{figure}
\end{frame}
\end{document}
